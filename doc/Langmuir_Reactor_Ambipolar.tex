%% LyX 2.3.6.1 created this file.  For more info, see http://www.lyx.org/.
%% Do not edit unless you really know what you are doing.
\documentclass[english]{article}
\usepackage[T1]{fontenc}
\usepackage[latin9]{inputenc}
\usepackage{babel}
\begin{document}

\subsubsection{Ambipolar Appoximation}

In the steady state where the background gas is dominant, we make
the congruence assumption that the flux of electrons and ions out
of any region must be equal, $\vec{\Gamma}_{e}=\vec{\Gamma}_{i}$,
such that charge does not build up. This is still true in the presence
of ionizing collisions, which create equal numbers of both negative
and positive species. Since the electrons are ligher, and would tend
to flow out faster (in an unmagnetized plasma), an electric field
must spring up to maintain the local flux balance. That is, a few
more electrons than ions initially leave the plasma region to set
up a charge imbalance and consequently an electric field. Let's expand
the flux balance, $\vec{\Gamma}_{e}=\vec{\Gamma}_{i}$,
\[
-\mu_{e}n_{e}\vec{E}-D_{e}\nabla n_{e}=\mu_{i}n_{i}\vec{E}-D_{i}\nabla n_{i}
\]
Note that the drift term is negative for electrons and positive for
ions, where E-field drags electrons down and speed ions up to balance
the fluxes. Assume charge neutrality in space, $n_{e}=n_{i}$, E-field
can be solved as
\[
\vec{E}_{ambi}=\vec{E}=\frac{D_{i}-D_{e}}{\mu_{i}+\mu_{e}}(\frac{\nabla n_{e}}{n_{e}})
\]
\[
\vec{E}_{ambi}=\vec{E}\approx-\frac{D_{e}}{\mu_{e}}(\frac{\nabla n_{e}}{n_{e}})
\]

This E-field is called ambipolar E-field and substituted to the flux,
\[
\vec{\Gamma}_{e,i}=\mu_{i}\frac{D_{i}-D_{e}}{\mu_{i}-\mu_{e}}\nabla n_{e}-D_{i}\nabla n_{e}=-\frac{\mu_{e}D_{i}+\mu_{i}D_{e}}{\mu_{i}+\mu_{e}}\nabla n_{e}
\]

You can see the coefficient is symmetric, and equal to electron and
ion. A new diffusion coefficient can be defined as
\[
D_{ambi}=\frac{\mu_{e}D_{i}+\mu_{i}D_{e}}{\mu_{i}+\mu_{e}}
\]
\[
\vec{\Gamma}_{e,i}=-D_{ambi}\nabla n_{e}
\]
\[
\mu_{e}=\frac{|q|}{v_{coll\_em}}(\frac{1}{m_{e}})\gg\mu_{i}=\frac{|q|}{v_{coll\_em}}(\frac{1}{m_{i}}),\:since\:m_{e}\ll m_{i}
\]
\[
D_{ambi}\approx D_{i}+\frac{\mu_{i}}{\mu_{e}}D_{e}=D_{i}(1+\frac{T_{e}}{T_{i}})
\]

put it back to the continuity equation,
\[
\frac{\partial n}{\partial t}+D_{ambi}\nabla^{2}n=S
\]

Now the continuity becomes a standard DIFFUSION equation with source
term, and much easier to solve. When using ambipolar diffusion appoximation,
we only calculate ion density $n_{i}$, and enforce charge neutrality,
$n_{e}=n_{i}$. Ambipolar E-field, $E_{ambi}$, can be used for electron
energy equation. In this way, Poisson's equation is avoided.

Computatinally, ion density, $n_{i}$, is first solved from the continuity
equation,
\[
\frac{\partial n_{i}}{\partial t}+D_{ambi}\nabla^{2}n_{i}=S_{i}
\]

After that, it is simply to put $n_{e}$ equal to $n_{i}$. And electric
field is obtained by,
\[
\vec{E}_{internal}=\vec{E}_{ambi}=\frac{D_{i}-D_{e}}{\mu_{i}+\mu_{e}}(\frac{\nabla n_{e}}{n_{e}})
\]

When there are more than one ions in plasmas, ambipolar assumption
might result in complex solution. Another trick could be imposed for
simplicity. Assume that each ion gets equilibrium only with electrons,
indicating that ion-ion interaction is ignored, we can have ambipolar
equations for each ion. Taking a simple case that contains two ions,
the original ambiplor assumption is,
\[
n_{e}=n_{i1}+n_{i2}
\]
\[
\vec{\Gamma}_{e}=\vec{\Gamma}_{i1}+\vec{\Gamma}_{i2}
\]

The stronger ambipolar assumption becomes,
\[
n_{e}=n_{e1}+n_{e2},\;n_{e1}=n_{i1},\;and\;n_{e2}=n_{i2}
\]
\[
\vec{\Gamma}_{e}=\vec{\Gamma}_{e1}+\vec{\Gamma}_{e2},\;\vec{\Gamma}_{e1}=\vec{\Gamma}_{i1},\;and\;\vec{\Gamma}_{e2}=\vec{\Gamma}_{i2}
\]

The additional assumption, in some degree, divides the whole plasma
into two independent plasmas. Ions interact only with electrons, instead
of other ions. Computationally, the densities of the two ions are
computed separately and independently. After that, make $n_{e}=n_{i1}+n_{i2}$.
Internal electric field is then only determined by the electron density.
\end{document}
